\documentclass{article}
    \usepackage{amsthm, amsmath}
    \usepackage{xeCJK}
    \usepackage{algorithm}
    \usepackage{algorithmicx}
    \usepackage{algpseudocode}
    \usepackage{hyperref}
    \usepackage{listings}
    
    \title{编译原理: PA3}
    \author{娄晨耀, 2016011343}
    \date{}
    
    
    \theoremstyle{plain}
    \newtheorem{thm}{Theorem}
    
    \theoremstyle{definition}
    \newtheorem{alg}{Algorithm}
    
    
    \begin{document}
    \maketitle

    \section{类的浅复制的支持}

    我通过在Trasnlator 里面新建一个函数genShallowCopy(Temp src) 来支持浅拷贝。这个函数会生成一段tac码,含义是新建一块和之前一样大的内存,然后逐个WORD拷贝,并把新建的内存地址返回。

    \section{sealed 的支持}

    sealed 是在编译时检查,不需要生成tac代码。

    \section{支持串行条件卫士}

    由于其语义相当于写多个if。于是实现起来和if一样,只是需要一个循环对每个条件都生成一下对应的tac代码即可。
    
    \section{支持简单的类型推导}

    这里其实在TypeCheck 阶段就可以推到出来类型。于是这里需要做的工作相当于VarDef + Assign。

    \section{数组操作}
    \subsection{数组初始化常量表达式}
    
    我这里直接利用了之前的genNewArray。之前genNewArray的实现是给全部值赋成0,我扩展了一下这个函数使得初始值可以传入。
    
    \subsection{数组下标动态访问表达式}

    这里判断一下其长度,然后直接选择是访问数字还是使用默认值即可。

    \subsection{数组迭代语句}

    这里像for一样,创建几个标签用来做控制。不同的是这里控制逻辑有点复杂,需要自己维护一下当前循环到数组的第几个元素了,同时也要判断一下while的条件是不是满足。

    支持var的话同之前一样,其在TypeCheck 阶段已经判断出其类型,不需要额外处理。

    break 的原理是把结束的标签加到一个栈里,一旦执行accept break 语句就会生成跳转到exit标签的语句。

    \section{动态检测除0}

    在Binary 生成除法操作之前,调用tr.genDivZeroCheck(right.val)。这是我新添加的一个辅助函数,会生成一段代码检测是否传入的值为0。

    \end{document}